% <-- a percent symbol indicates a comment which does not affect the output of LaTeX
% you can leave the preamble alone, from here ...
\documentclass[12pt]{article}

\usepackage{amssymb,amsmath,amsthm}
\usepackage[top=1in, bottom=1in, left=1.25in, right=1.25in]{geometry}
\usepackage{enumitem,palatino}
\usepackage[final]{graphicx}
\usepackage[colorlinks=true,citecolor=blue,linkcolor=red,urlcolor=blue]{hyperref}

\newtheorem{problem}{Problem}
% ... to here

% shortcuts for blackboard bold number sets (reals, integers, etc.)
\newcommand{\II}{\ensuremath{\mathbb I}}
\newcommand{\NN}{\ensuremath{\mathbb N}}
\newcommand{\QQ}{\ensuremath{\mathbb Q}}
\newcommand{\RR}{\ensuremath{\mathbb R}}
\newcommand{\ZZ}{\ensuremath{\mathbb Z}}

\newcommand{\eps}{\ensuremath{\epsilon}}
\newcommand{\ds}{\displaystyle}

% feel free to add more shortcuts here


\begin{document}
% replace with your name, but otherwise leave this header alone, from here ...
\small
\noindent \textsc{Math F307: Homework Assignment 1} \hfill Christopher Munoz

\normalsize
\bigskip
% ... to here
\section*{Section 1.1}
\setcounter{problem}{3}
\begin{problem}
Give the value of each of the following.
\begin{enumerate}[label=(\alph*)]
    \item $\lceil 0.763 \rceil$
    \item $2\lceil 0.6 \rceil - \lceil 1.2 \rceil$
    \item $\lceil 1.1 \rceil + \lceil 3.3 \rceil$
    \item $\lfloor \sqrt{3} \rfloor - \lfloor \sqrt{3} \rfloor$ 
    \item $\lceil -73 \rceil - \lfloor -73 \rfloor$ 
\end{enumerate}
\end{problem}

\begin{proof}[Solution]
~
\begin{enumerate}[label=(\alph*)]
    \item 
      $ \lceil 0.763 \rceil = (1) = 1 $
    \item 
      $2\lceil 0.6 \rceil - \lceil 1.2 \rceil = 2(1) - (2) = 0$
    \item
      $\lceil 1.1 \rceil + \lceil 3.3 \rceil = (2) + (4) = 6$

    \item 
      $\lfloor \sqrt{3} \rfloor - \lfloor \sqrt{3} \rfloor = (2) - (1) = 0$ 
    \item
      $\lceil -73 \rceil - \lfloor -73 \rfloor = (-73) - (-73) = 0$ 

\end{enumerate}
\end{proof}


\setcounter{problem}{7}
\begin{problem}
How many multiples of 10 are there between the following pairs of numbers?
\begin{enumerate}[label=(\alph*)]
    \item 1 and 80
    \item 0 and 100
    \item 9 and 2967
    \item $-6$ and 34
    \item $10^4$ and $10^5$
    \item $-600$ and 3400
\end{enumerate}
\end{problem}

\begin{proof}[Solution]
  (Assuming inclusive)
\begin{enumerate}[label=(\alph*)]
    \item 
      $8$
    \item 
      $\lfloor \frac{n}{k} \rfloor - \lfloor \frac{m-1}{k} \rfloor = \lfloor \frac{100}{10} \rfloor - \lfloor \frac{-1}{10} \rfloor = 11$
    \item 
      $\lfloor \frac{2967}{10} \rfloor - \lfloor \frac{8}{10} \rfloor = 296$
    \item 
      $\lfloor \frac{34}{10} \rfloor - \lfloor \frac{-7}{10} \rfloor = 4$
    \item 
      $\lfloor \frac{10,000}{10} \rfloor - \lfloor \frac{99,999}{10} \rfloor = 9001$
    \item 
      $\lfloor \frac{3400}{10} \rfloor - \lfloor \frac{-600}{10} \rfloor = 401$
\end{enumerate}
\end{proof}tw+

\setcounter{problem}{17}
\begin{problem}
\begin{enumerate}[label=(\alph*)] 
    \item What does Fact 4 say for $k = 1$? Is this statement obvious?
    \item What does Fact 4 say for $k > n$? Is this statement obvious?
\end{enumerate}
\end{problem}

\begin{proof}[Solution]
~
\begin{enumerate}[label=(\alph*)]
    \item That $\lfloor n/1 \rfloor = n$ for all $n \in \ZZ$. Obvious, every positive integer is a multiple of 1 and everyone naturally knows there are $n$ numbers between $1$ and $n$. 
    \item For $ k > n$, $\lfloor n/k \rfloor = 0$, Obvious, follows from the fact that if $k > n$ then $0 \leq n/k < 1$, the floor of this range is $0$ always.
\end{enumerate}
\end{proof}

\setcounter{problem}{18}
\begin{problem}
\begin{enumerate}[label=(\alph*)]
    \item Give a specific example of numbers $x$ and $y$ for which $\lfloor x \rfloor + \lfloor y \rfloor < \lfloor x + y \rfloor$.
    
    \item Give a specific example of numbers $x$ and $y$ for which $\lfloor x \rfloor + \lfloor y \rfloor = \lfloor x + y \rfloor$.
    
    \item Give a convincing argument that $\lfloor x \rfloor + \lfloor y \rfloor \leq \lfloor x + y \rfloor$ for every pair of numbers $x$ and $y$. 
    
    \textit{Suggestion: Use the fact that $\lfloor x + y \rfloor$ is the largest integer less than or equal to $x + y$.}
\end{enumerate}
\end{problem}

\begin{proof}[Solution]
~
\begin{enumerate}[label=(\alph*)]
    \item Let $x = y = 1.5$, then  $\lfloor x \rfloor + \lfloor y \rfloor < \lfloor x + y \rfloor =  \lfloor 1.5 \rfloor + \lfloor 1.5 \rfloor < \lfloor 1.5 + 1.5 \rfloor = 2 < 3$.
    \item Let $x = y = 1$, then $\lfloor x \rfloor + \lfloor y \rfloor = \lfloor x + y \rfloor = \lfloor 1 \rfloor + \lfloor 1 \rfloor = \lfloor 1 + 1 \rfloor = 2$.
    \item Proof: Let $x, y \in \RR$. From the definition of the floor function we have $\lfloor x \rfloor \leq x$ and $\lfloor y \rfloor \leq y$. Adding these two inequalities together gives
      $$\lfloor x \rfloor + \lfloor y \rfloor \leq x + y.$$
      Note that $\lfloor x \rfloor + \lfloor y \rfloor$ is an integer (since it's the sum of two integers). Given that $\lfloor x + y \rfloor$ is the largest integer less than or equal to $x + y$, we have 
      $$\lfloor x + y \rfloor \leq x + y.$$
      Since $\lfloor x \rfloor + \lfloor y \rfloor$ is an integer satisfying $\lfloor x \rfloor + \lfloor y \rfloor \leq x + y$, and $\lfloor x + y \rfloor$ is the \emph{largest} such integer, we must have
     $$\lfloor x \rfloor + \lfloor y \rfloor \leq \lfloor x + y \rfloor.$$
 \end{enumerate}
\end{proof}

\section*{Section 1.2}

\setcounter{problem}{1}
\begin{problem}
True or False. Explain briefly.
\begin{enumerate}[label=(\alph*)]
    \item $n \mid 1$ for all positive integers $n$.
    
    \item $n \mid n$ for all positive integers $n$.
    
    \item $n \mid n^2$ for all positive integers $n$.
\end{enumerate}
\end{problem}

\begin{proof}[Solution]
~
\begin{enumerate}[label=(\alph*)]
\item \textbf{False.} For the statement $n \mid 1$ (equivalently $1 = nk$) to hold, there would need to be an integer $k$ such that $1 = nk$. By the definition of divisibility, $k$ must be an integer. For $n = 2$, we would need $k = 1/2$, which is not an integer. In fact, only $n = 1$ satisfies this (with $k = 1$). Therefore, the statement is false for all positive integers $n > 1$.
  \item \textbf{True.} $n \mid n$ is true for all positive integers $n$. This is equivalent to $n = nk$ for some integer $k$. Taking $k = 1$ gives $n = n \cdot 1$, which is always true.
  \item \textbf{True.} $n \mid n^2$ for all positive integers $n$ is true. Observe that if we let $k = n$ in the equation $n^2 = nk$, we have $n^2 = n \cdot n$, which is always true.
\end{enumerate}
\end{proof}

\setcounter{problem}{13}
\begin{problem}
Suppose that $m$ and $n$ are integers that are multiples of $d$, say $m = ad$ and $n = bd$.
\begin{enumerate}[label=(\alph*)]
    \item Explain why $d \mid lm$ for every integer $l$.
    
    \item Show that $m + n$ and $m - n$ are multiples of $d$.
    
    \item Must $d$ divide $17m - 72n$? Explain.
\end{enumerate}
\end{problem}

\begin{proof}[Solution]
~
\begin{enumerate}[label=(\alph*)]
  \item Since $ m = ad$, that means $lm = l(ad) = (la)d$. By definition of divisiblity this means $d \mid lm$.
    \item Since $m = ad$ and $n = bd$, it follows that $m + n = ad + bd = (a+b)d$. By definition of divisibility this means $d \mid m + n$. Similarly
      $m - n = ad - bd = (a-b)d$ so $d \mid m - n$.
    \item Yes, because $17m - 72n = 17(ad) - 72(bd) = 17ad-72bd = (17a - 72b)d$. This means $d \mid 17m - 72n$ by definition.

\end{enumerate}
\end{proof}


\setcounter{problem}{15}
\begin{problem}
\begin{enumerate}[label=(\alph*)]
    \setcounter{enumi}{1}
    \item List the positive integers less than 36 that are relatively prime to 36.
\end{enumerate}
\end{problem}

\begin{proof}[Solution]
A positive integer $n$ is relatively prime to $36$ if and only if $\gcd(n,36)=1$ holds which means $n$ shares no prime factors with $36$.
Here is our list of positive integers less than $36$ that have this property:
$$1,5,7,11,13,17,19,23,25,29,31,35$$
\end{proof}

\section*{Section 1.3}

\setcounter{problem}{1}
\begin{problem}
List the elements in the following sets.
\begin{enumerate}[label=(\alph*)]
    \item $\{1/n : n = 1, 2, 3, 4\}$
    
    \item $\{n^2 - n : n = 0, 1, 2, 3, 4\}$
    
    \item $\{1/n^2 : n \in \mathbb{P}, \text{ $n$ is even and } n < 11\}$
    
    \item $\{2 + (-1)^n : n \in \NN\}$
\end{enumerate}
\end{problem}

\begin{proof}[Solution]
~
\begin{enumerate}[label=(\alph*)]
  \item  $\{1/n : n = 1, 2, 3, 4\} = \{1, \frac{1}{2}, \frac{1}{3}, \frac{1}{4}\}$

    
  \item  $\{n^2 - n : n = 0, 1, 2, 3, 4\} = \{0, 1, 2, 6, 12\}$

    
  \item  $\{1/n^2 : n \in \mathbb{P}, \text{ $n$ is even and } n < 11\} = \{\frac{1}{4}\}$ ($n = 2$ is only prime even)


    
  \item $\{2 + (-1)^n : n \in \NN\} = \{1 ,3 \}$ 

\end{enumerate}
\end{proof}

\setcounter{problem}{5}
\begin{problem}
Repeat Exercise 4 for the following sets.
\begin{enumerate}[label=(\alph*)]
    \item $\{n \in \NN : n \mid 12\}$
    
    \item $\{n \in \NN : n^2 + 1 = 0\}$
    
    \item $\{n \in \ZZ : \lfloor \frac{n}{3} \rfloor = 8\}$
    
    \item $\{n \in \NN : \lceil \frac{n}{2} \rceil = 8\}$
\end{enumerate}
\end{problem}

\begin{proof}[Solution]
~
\begin{enumerate}[label=(\alph*)]
  \item  $\{n \in \NN : n \mid 12\} = \{1, 2, 3, 4, 6, 12\}$ 
    \item $\{n \in \NN : n^2 + 1 = 0\} = \emptyset$ 
    \item   $\{n \in \ZZ : \lfloor \frac{n}{3} \rfloor = 8\} = \{24, 25, 26\}$
    \item  $\{n \in \NN : \lceil \frac{n}{2} \rceil = 8\} = \{15, 16\}$


\end{enumerate}
\end{proof}

\setcounter{problem}{7}
\begin{problem}
How many elements are there in the following sets? Write $\infty$ if the set is infinite.
\begin{enumerate}[label=(\alph*)]
    \item $\{n \in \NN : n^2 = 2\}$
    
    \item $\{n \in \ZZ : 0 \leq n \leq 73\}$
    
    \item $\{n \in \ZZ : 5 \leq |n| \leq 73\}$
    
    \item $\{n \in \ZZ : 5 < n < 73\}$
    
    \item $\{n \in \ZZ : \text{$n$ is even and } |n| \leq 73\}$
    
    \item $\{x \in \QQ : 0 \leq x \leq 73\}$
    
    \item $\{x \in \QQ : x^2 = 2\}$
    
    \item $\{x \in \RR : x^2 = 2\}$
\end{enumerate}
\end{problem}

\begin{proof}[Solution]
~
\begin{enumerate}[label=(\alph*)]
    \item  Cardinality of 0    
    \item  Cardinality of 74
    \item  Cardinality of 2(73 - 5) = 136
    \item  Cardinality of 73-5 = 68
    \item  Cardinality of $\lfloor \frac{73}{2} \rfloor - \lfloor \frac{-73}{2} \rfloor$ = 73
    \item Cardinality of $\infty$, specifically $\aleph_0$ (countably infinite). 
    \item Cardinality of 0 
    \item Cardinality of 2

\end{enumerate}
\end{proof}



\end{document}
